Le projet sera fait selon une approche itérative basée sur Scrum. Dans le vocabulaire de Scrum, le projet sera considéré comme un Produit. Ce produit aura un backlog, c'est-à-dire un ensemble de fonctionnalités non implémentées. Chaque itération aura un but en lien avec le but général énoncé ci-haut. Pour atteindre ce but, un ensemble d'items du backlog sera sélectionné. Durant chaque itération, l'ensemble des activités – analyse, conception, implémentation, tests , etc. – liées au développement seront accomplies : l'objectif ici est de fournir à chaque itération une ou plusieurs fonctionnalités terminées. En ce sens, chacune des itérations doit fournir une fonctionnalité complète et cohérente : en théorie, le projet devrait pouvoir être livré à la fin de chaque itération sans qu'il y ait de fonctionnalité non implémentée. À chaque fin d'itération, le travail accompli sera évalué.
