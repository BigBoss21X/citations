\subsection{Description des artéfacts}
    \begin{tabular}{| p{4cm} | p{12.75cm} |}
      \hline
        \rowcolor[gray]{.9}
        Nom de l'artéfact & Description \\
        \hline
        Ensemble de test & Fichiers texte et \emph{boxfile} utilisés pour évaluer le prototype \\
        \hline
        Résultats des tests & Documents XML décrivant les résultats des tests à chaque étape \\ 
        \hline
        Code source & Code source du prototype \\
        \hline
        Manuel de l'utilisateur & Description des fonctionnalités principale, guide d'installation et programmation \\
        \hline
        Archive & Scans de l'archive de \emph{Société} \\
      \hline
    \end{tabular}
\subsection{Planification}
\section{Risques}
\begin{tabular}{| p{2.75cm} | p{3.75cm} | p{3.75cm} | p{5.75cm} |}
      \hline
    \rowcolor[gray]{.9} 
    Risque & Impact & Probabilité & Mitigation / atténuation \\
    \hline
    Objectifs mal définis & Projet en retard / peu fonctionnel & Moyenne & Utiliser une approche itérative et contrôler la progression régulièrement \\
    \hline
    Difficulté d'implémentation & Moyen & Moyenne & Acquérir une base théorique solide avant de débuter \\
    \hline
    Dépendance externe pour l'obtention du document visé par la référence & Moyen & Élevée & Prototypage rapide afin de savoir rapidement s'il est possible (et simple) d'obtenir les références \\
      \hline
\end{tabular}

