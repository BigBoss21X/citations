L'identification des sources et la recherche de références bibliographiques sont à la base de la recherche académique. Tout lecteur désirant approfondir les connaissances qu'il a de son domaine d'intérêt ira consulter les références bibliographiques des textes qu'il lit pour pouvoir obtenir les références connexes.

Les publications universitaires prennent essentiellement le même format, à quelques différences près selon la discipline. Cette standardisation du format leur permet de véhiculer le message plus efficacement en tirant profit de l'association cognitive qui se crée implicitement chez le lecteur entre le format du texte et sa fonction. 

Par exemple, le format de la citation varie très peu d'un article scientifique à l'autre et cette variation est dans la majorité des cas explicable par le standard particulier qui a été adopté dans une discipline donnée. La citation est généralement introduite selon la méthode classique ou la méthode auteur-date. Pour la méthode classique, le chiffre identifiant la citation est en exposant et la référence complète se trouve dans la note de bas de page sous la forme: \emph{nom, prénom, titre de l'ouvrage, ville, maison d'édition, année de publication, collection, page}. Pour la méthode auteur-date, la citation est insérée directement dans le texte et prends la forme \emph{(Auteur, année, page)}. La référence complète est disponible à la fin du document dans une bibliographie. Ainsi ce type de codification permet d'exprimer la majorité des citations.

Par ailleurs, les récents développements de la technique ont permis la création de bibliothèques numériques exhaustive. Par exemple le \emph{Library Project} de \emph{Google} vise à numériser le contenu de plusieurs bibliothèques majeures \footnote{Pour connaître la liste des bibliothèques partenaires du projet, voir http://books.google.com/googlebooks/partners.html}. Cette interface uniforme vers le contenu de plusieurs bibliothèques permet d'avoir un accès facile et standardisé à un grand ensemble de ressources bibliographiques. En d'autres termes, les outils comme \emph{Google Books} vient standardiser l'autre moment de la recherche bibliographique, c'est-à-dire la manière d'accéder à la ressource qui est identifiée par la référence.

Comment peut-on tirer profit des bibliothèques numériques existantes pour automatiser l'accès à la ressource visée par la référence d'un texte? Avec quelle précision est-il possible d'obtenir la version numérique d'un texte visé par une référence bibliographique? Le projet évaluera l'hypothèse selon laquelle l'information contenue dans la mise en forme des citations et références est suffisamment uniforme pour être codifiable dans un ensemble restreint de règles. Les manières d'obtenir les références sur les bibliothèques virtuelles seront ensuite explorées.

