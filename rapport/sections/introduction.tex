De nombreux projets visent à numériser des références bibliographiques traditionnelles. Les références numériques ont ceci d'intéressant par rapport aux références traditionnelles qu'il est possible d'utiliser l'ordinateur pour faire des analyses sur un corpus dont la taille aurait rendu la tâche impossible autrement.

Les références numériques offrent aussi, en raison de leur format, une plus grande interactivité, car en plus de contenir le texte de la référence -- leur vocation première, elles peuvent aussi contenir, et ce de manière transparente pour l'utilisateur, un ensemble de métadonnées. Ces métadonnées peuvent alors servir de support sur lequel sont basées les fonctionnalités offertes par le document.

Dans ce texte, nous allons analyser les manières d'augmenter les références d'une ressource référence numérisée en faisant pointer les références textuelles vers l'endroit où réside le texte sur Internet. Nous allons dans un premier temps décrire la problématique propre à l'identification de références et à la localisation de celles-ci, pour ensuite décrire le travail qui sera accompli pour résoudre le problème. Après avoir décrit les objectifs, la méthode retenue pour accomplir le travail sera décrite et la description de la solution sera offerte.
