À la suite de la réalisation du prototype, il est permis de conclure qu'il serait possible de poursuivre dans cette voie et de compléter le produit. Bien que certaines modifications soient requises pour passer du stade de prototype au stade de produit fini, la base sur laquelle a été fondé le prototype.

Plusieurs hypothèses ont été posées: certaines ont été confirmées tandis que certaines ont été infirmées. D'abord, il est possible de construire un système expert pour faire l'identification des appels de notes de bas de page. Les règles sont connues \emph{a priori} et il est possible de les codifier: il faut simplement un modèle flexible qui est capable de fournir l'information nécessaire (par exemple la taille moyenne d'une lettre précise du texte)

Il est par contre impossible de présumer que les références bibliographiques citées dans le texte peuvent être identifiées «naïvement» en utilisant des expressions régulières.

En outre, si l'application que nous avons retenue pour le développement du prototype a été l'extraction de ressources bibliographiques, il faut noter qu'il en existe d'autres qui pourraient être intéressantes à explorer. Plus précisément, il serait possible d'utiliser le modèle et le module d'identification de références bibliographiques pour étudier les réseaux au sein des publications académiques en établissant une cartographie des citations. Une fois les citations identifiées et étiquetées, il est possible de construire un index de références et d'établir ainsi les liens entre les publications.
