Le but premier d'un prototype est de vérifier si une hypothèse de développement peut être fructueuse et déterminer s'il est souhaitable de poursuivre dans cette voie. Le prototype qui a été implémenté fonctionne bien, hormis dans le cas où la référence bibliographique est précédée de texte libre, mais nécessiterait encore beaucoup de travail pour devenir un produit fini.

En premier lieu, il faut noter que les différents modules sont fonctionnels lorsqu'ils sont exécutés indépendamment les uns des autres, mais que les défis propres à leur intégration n'ont pas encore été résolus. La raison principale est que l'une des hypothèses de travail s'est avérée être fausse: on ne peut pas dire qu'il y a identité entre une référence bibliographique et une note de bas de page.
L'implémentation de \emph{anystyle-parser} est efficace pour étiqueter les parties de la référence bibliographique, mais est incapable d'identifier une «non-référence»: elle prend pour acquis qu'elle aura une référence complète, ce qui est pratiquement jamais le cas dans le domaine de ce projet. Il devient donc nécessaire d'implémenter une librairie qui fait l'extraction de la référence bibliographique dans un document rédigé en texte libre et qui deviendrait l'étape préalable à l'étiquetage. Cette librairie devra aussi être capable de gérer les références symboliques dans le texte (par exemple l'utilisation de \emph{ibid}, pour référer à la dernière citation, \emph{op cit.}, pour référer au texte cité plus haut, etc.)


La librairie \emph{anystyle-parser} devra ensuite être entraînée pour fonctionner avec des références en français. Il est raisonnable de penser qu'il est possible d'obtenir, avec un modèle correctement entraîné, des résultats similaires à ceux obtenus avec des références bibliographiques en anglais.

Par ailleurs, la performance de \emph{anystyle-parser} est médiocre dans le prototype. L'implémentation du prototype est en python tandis que celle de cette librairie est en ruby: le prototype lance l'interpréteur ruby à chaque fois qu'il doit analyser une référence bibliographique, ce qui prends à chaque fois quelques secondes.

Pour résoudre ce problème, la meilleure solution serait à notre avis d'exécuter la librairie comme un service web et de communiquer avec elle dans un format intermédiaire, par exemple JSON. De cette manière, l'interpréteur ruby ne serait lancé qu'une seule fois. Il s'agit d'une solution pragmatique: il serait par ailleurs moins compliqué d'implémenter ce service web avec un serveur léger comme \emph{Sinatra}~\cite{sinatrarb} que de réimplémenter \emph{anystyle-parser} en python.

Du point de vue de la conception, des modifications doivent être apportées pour finir de découpler les modules. Le modèle, tel qu'implémenté dans le prototype, peut apporter de sérieuses limitations à long terme. Dans le prototype, il y a confusion entre les informations concernant la structure du document (les pages, les mots, etc.) et l'information sémantique (les appels et les références bibliographiques) du document. Les deux niveaux sont implémentés dans les mêmes objets, et les informations sémantiques sont obligatoirement rattachées à un de ces objets du modèle: par exemple, autant l'appel de note de bas de page que la citation doit être rattachée à une page.

Pour faire évoluer le prototype et être capable de travailler avec d'autres systèmes de notation des appels (où les références sont toutes regroupées en fin de chapitre ou en fin de document, par exemple), il serait important de poser cette distinction entre le niveau structurel et le niveau sémantique. Une solution pourrait être d'implémenter un système d'annotation sémantique qui agirait sur un modèle qui serait dépouillé de tout sauf les informations structurales. Le modèle pourrait être réutilisé plus facilement pour d'autres applications -- notamment la simple correction de numérisations.

Le prototype a été conçu pour faire la recherche sur Google Scholar. Il s'acquite bien de cette tâche. Il a cependant été conçu selon l'hypothèse que les bibiliothèques virtuelles seraient similaires. Après vérification, cette hypothèse s'est avérée être fausse: chaque bibliothèque requiert sa propre implémentation. Pour Google Scholar et arXiv, on peut utiliser leur API \emph{ATOM} pour accéder aux ressources. Certaines bibliothèques n'offrent pas ces ressources et on doit alors faire du \emph{screen scraping}\footnote{Ou en français « Capture de données d'écran». Il s'agit de récupérer les informations qui sont destinées au navigateur web.}.
